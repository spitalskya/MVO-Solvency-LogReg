\documentclass[report.tex]{subfiles}

\begin{document}
	
	\section{Záver a diskusia}	
	
	V našom projekte sme sa venovali matematickej formulácii a implementácii lineárnej regresie minimalizovaním $L^1$ a $L^{\infty}$ noriem. Vizualizovali sme funkčnosť implementácie na dátach \verb|A04plotregres.npz|. Pre dáta \verb|A04wine.csv| sme regresiou predikovali budúcu cenu vína a zisťovali sme, ktoré parametre na ňu najviac vplývajú. Takisto sme pre túto predikciu spočítali $R^2$ koeficient, ktorý ukázal relatívnu vhodnosť nášho modelu. Nakoniec sme predstavili implementáciu týchto regresných modelov v jazyku \verb|Python| pre ľubovoľné číselné dáta a mierne sme analyzovali správanie sa jednotlivých modelov. Nakoniec sme aj sformulovali a implementovali model pre minimalizovanie váženej sumy noriem.
	
	Myslíme si, že naše modely sú jednoduchým nástrojom pre počítanie lineárnej regresie. Ako ďalšie pokračovanie projektu by sme mohli skúmať charakteristiky jednotlivých modelov a zistiť, pre aké dáta je lepšie použiť jednotlivé normy. Tiež by sme sa mohli zaoberať ich časovou komplexitou (napríklad aj v porovnaní s $L^2$ lineárnou regresiou) a všeobecnou interpretáciou výsledných $\beta$ koeficientov pre oba prístupy.
	
\end{document}