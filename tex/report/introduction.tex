\documentclass[report.tex]{subfiles}

\begin{document}
	
\section{Predstavenie témy}	

V našom projekte sa budeme venovať vytváraniu modelu na binárnu klasifikáciu dát. Na pozadí tohto modelu prebieha minimalizácia konkrétnej účelovej funkcie pomocou kvázinewtonovských alebo gradientných metód. 

Na začiatku si odvodíme účelovú funkciu, odôvodníme jej tvar a odvodíme jej gradient. Následne ju budeme minimalizovať pomocou kvázinewtonovských a gradientných metód s rôznou dĺžkou kroku a vizualizujeme ich konvergenciu. 

Taktiež otestujeme funkčnosť modelu na poskytnutých dátach o solventnosti klientov. O každom klientovi máme uvedené jeho počet mesiacov od otvorenia účtu, pomer úspor a investícií, počet rokov v súčasnom zamestnaní a binárny údaj, či je klient solventný (1) alebo nie (0). My sa budeme snažiť predikovať solventnosť klienta na základe prvých troch údajov. 

\subsection{Zavedenie značenia}

\begin{itemize}
	\item $m = 699$ značí počet klientov, o ktorých máme dáta
	\item $v \in \mathbb{R}^m$, $i$-ta zložka má hodnotu 1, ak je klient $i$ solventný, inak 0
	\item $u_j \in \mathbb{R}^m$, $j=1,2,3$, vektory údajov o klientoch
	\begin{itemize}
		\item[$\circ$] $u_1$ -- počet mesiacov od otvorenia účtu
		\item[$\circ$] $u_2$ -- pomer úspor a investícií
		\item[$\circ$] $u_3$ -- počet rokov v súčasnom zamestnaní
	\end{itemize}
	\item $v^i,u_j^i$ označujú $i$-te položky jednotlivých vektorov pre $i=1,\dots,m$, $j=1,2,3$
\end{itemize}
	
\end{document}