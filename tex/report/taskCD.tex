\documentclass[report.tex]{subfiles}

\begin{document}   

\section{Riešenie optimalizačnej úlohy}\label{sec:CD}

V tejto časti sa venujeme riešeniu optimalizačnej úlohy \ref{eq:objf} rôznymi metódami. Tie boli implementované v Pythone. Konkrétne sme implementovali gradientné metódy (s optimálnou a konštnantou dĺžkou kroku) a kvázinewtonovské metódy BFGS a DFP (s približne optimálnou dĺžkou kroku nájdenou backtracking-om alebo s optimálnou dĺžkou kroku, nájdenou bisekciou). 

Ako štartovací bod sme pri každej metóde volili $x_0 = (0,0,0,0)^T$ a ako kritérium optimality bolo použité $||\nabla J(x^k)|| \leq 10^{-3}$. Optimálnym bodom bude teda vektor parametrov $x$, ktorý budeme používať v logistickej funkcii na odhadovanie solventnosti klienta podľa jeho dát.
	

\subsection{Kvázinewtonovské metódy}

Minimalizujeme \ref{eq:objf} pomocout metód BFGS a DFP s optimálnym krokom (nájdeným bisekciou) a približne optimálnym krokom (nájdeným backtrackingom).

\begin{center}
	\small
	\begin{tabular}{| c | c  c  c  c |}
		\hline
		 & $x_0$ & $x_1$ & $x_2$ & $x_3$ \\
		\hline
		BFGS + backtracking & 0.20751015 & -0.04712048 & 0.31535175 & 0.30654686 \\
		BFGS + bisekcia & 0.20751337 & -0.04712051 & 0.31535088 & 0.30654664 \\
		DFP + backtracking & 0.20750999 & -0.04712047 & 0.31535176 & 0.30654688 \\
		DFP + bisekcia & 0.20751338 & -0.04712052 & 0.31535087 & 0.30654663 \\
		\hline
	\end{tabular}
\end{center}

Všetky štyri minimalizácie skonvergovali k minimu (vzhľadom na kritérium optimality) za menej ako 13 iterácií. Vidíme, že optimálne hodnoty všetkých štyroch minimalizácií sa odlišujú najskôr v ráde $10^{-5}$, čiže môžeme predpokladať, že konvergujú k rovnakému bodu minima.

Môžeme si všimnúť, že pozitívny vplyv na pravdepodobnosť solventnosti klienta má druhý sledovaný parameter, čiže pomer úspor a investícií, a tretí sledovaný porameter, čiže počet rokov v súčasnom zamestnaní. Takisto si môžeme všimnúť, že počet mesiacov od otvorenia účtu (prvý sledovaný parameter), má na odhad pravdepodobnosti solventnosti klienta negatívny vplyv, čo je prekvapivý výsledok. 

Nájdený koeficient $x_0$ má za následok to, že pre klienta, ktorého parametre sú $(0,0,0)^T$, po dosadení do logistickej funkcie dostaneme pravdepodobnosť solventnosti približne 0.5517.

Môžeme si takisto porovnať čas (v sekundách) potrebných na nájdenie minima pre jednotlivé metódy.

\begin{center}
	\small
	\begin{tabular}{| c | c |}
		\hline
		& čas[s]  \\
		\hline
		BFGS + backtracking & 0.0067  \\
		BFGS + bisekcia & 0.0074  \\
		DFP + backtracking & 0.0031  \\
		DFP + bisekcia & 0.0069  \\
		\hline
	\end{tabular}
\end{center}

Vidíme, že pre obidve implementované kvázinewtonovské metódy je implementácia s približne optimálnou dĺžkou rýchlejšia.

\subsection{Gradientné metódy}

Podobne ako vyššie, minimalizujeme \ref{eq:objf} pomocou gradientnej metódy s optimálnym a s konštantným krokom. Na nájdenie optimálneho kroku používame bisekciu, ako konštantný krok používame $2\cdot 10^{-5}$.

\begin{center}
	\small
	\begin{tabular}{| c | c  c  c  c |}
		\hline
		& $x_0$ & $x_1$ & $x_2$ & $x_3$ \\
		\hline
		optimálny krok & 0.20742273 & -0.04711977 & 0.31535679 & 0.30656397 \\
		konštantný krok & 0.19322267 & -0.0470058 &  0.31617533 & 0.30934507 \\
		\hline
	\end{tabular}
\end{center}

Gradientná metóda s optimálnym krokom skonvergovala (vzhľadom na kritérium optimality) po rádovo 5000 iteráciách. Gradientná metóda s konštantným krokom nedokonvergovala (vzhľadom na kritérium optimality) ani po 10000 iteráciách (neboli sme experimentovaním nájsť vhodnú dĺžku kroku). 

Signifikancia jednotlivých parametrov klientov je zhodná s tou, ktorá je popísaná vyššie. Takisto, pre klienta $(0,0,0)^T$ je odhad pravdepodobnosti solventnosti približne 0.5517 (pre optimálny krok), resp. 0.5482 (pre konštantný krok).

Rovnako ako vyššie, môžeme porovnať časy potrebné na minimalizáciu.

\begin{center}
	\small
	\begin{tabular}{| c | c |}
		\hline
		& čas[s]  \\
		\hline
		optimálny krok & 6.1772  \\
		konštantný krok & 0.8142  \\
		\hline
	\end{tabular}
\end{center}

Vidíme, že hľadanie optimálneho kroku v každej iterácií pridá približne 5.3 sekundy k času výpočtu, aj keď iterácií bolo rádovo polovica oproti konštantnému kroku. Skúsili sme preto nastaviť maximálny počet iterácií pre gradientnú metódu s konštatným krokom na $10^5$. Už po rádovo 30000 iteráciách bolo dosiahnuté kritérium optimality a jeho nájdenie trvalo približne 0.9181 sekundy. Vidíme teda, že pri vysokom počte iterácií môže mať zmysel použiť skôr konštatný krok, keďže vieme výrazne ušetriť čas potrebný na hľadanie optimálneho kroku. 

\begin{equation*}
	J^*_{GM~const} = (0.20742016,~-0.04711976,~0.31535694,~0.30656448)^T
\end{equation*}


\end{document}
