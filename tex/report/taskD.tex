\documentclass[report.tex]{subfiles}


\begin{document}   

\section{Počítanie $R^2$}\label{sec:D}
Teraz sa budeme venovať počítaniu koeficientu determinácie (tzv. $R^2$ koeficient) pre dáta a lineárnu regresiu z predošlej sekcie. $R^2$ koeficient hovorí o tom, aký podiel variability závislej premennej model zachytáva. Hodnoty v blízkosti 1 naznačujú \enquote{lepší} model.

Vytvorme funkciu \pyth|r_squared(x, y, beta)|, kde \pyth|x| je matica vektorov nezávislých premenných, \pyth|y| je vektor závislej premennej, \pyth|beta| je vektor optimálnych $\beta$ koeficientov získaných lineárnou regresiou, ktorá bude počítať $R^2$ koeficient podľa definície:

\begin{align*}
	&R^2 = 1 - \frac{\sum_{i=1}^{n} (y_i - \hat{y}_i)^2}{\sum_{i=1}^{n} (y_i - \bar{y})^2} &\hat{y} = \beta_0 + \beta_1x_1 + \dots + \beta_kx_k,~\bar{y} = \frac{1}{n} \sum_{i=1}^ny_i
\end{align*}

\begin{python}
def r_squared(x: np.ndarray, y: np.ndarray, beta: np.ndarray) -> float:
	# calculate y-hat and mean of y vector
	y_hat = beta[0] + x @ beta[1:]
	y_mean = np.mean(y)
	
	res1 = 0    # partial result for the numerator in the formula
	res2 = 0    # partial result for the denominator in the formula
	
	# calculate the sums
	res1 = np.sum((y - y_hat)**2)
	res2 = np.sum((y - y_mean)**2)
	
	# calculate the R^2 coefficient
	result = 1 - (res1 / res2)
	return result
\end{python}

Implementujeme funkciu na dátach \verb|A04wine.csv|. Načítame dáta, rozdelíme ich do jednotlivých premenných, vyriešime potrebné LP problémy (rovnako ako v predošlej úlohe) a vypočítame $R^2$ koeficient:

\begin{python}
	betas = solve.x[:k+1]
	betas_inf = solve_inf.x[:k+1]
	
	r_squared(x, y, betas)
	r_squared(x, y, betas_inf)
\end{python}
	
Vypočítané príslušné koeficienty determinácie teda sú:

\begin{align*}
	R^{2}_{(1)} &\approx  0.78813 \\
	R^{2}_{(\infty)} &\approx 0.80649
\end{align*}

Z toho môžeme usúdiť, že náš model sa dá považovať za relatívne vhodný pre tieto dáta. Tiež vidíme, že lineárna regresia pomocou Chebyshevovej normy lepšie zachytáva rozptyl dát.

\end{document}
