\documentclass[report.tex]{subfiles}

\begin{document}
\section{Odvodenie účelovej funkcie a jej gradientu}\label{sec:AB}

V tejto časti sa budeme venovať odvodzovaniu účelovej funkcie a jej gradientu, ktorú v neskorších častiach budeme minimalizovať, pomocou čoho vytvoríme model na binárnu klasifikáciu. 

Do logistickej funkcie $g(z) = \frac{1}{1+e^{-z}}$, ktorá bude odhadovať pravdepodobnosť solventnosti klienta, budeme dosádzať hodnoty $z=x^Tu^i$ pre vektor parametrov $x = (x_0,\dots,x_3)$ a vektor údajov o klientovi $u^i = (1, u_1^i, u_2^i, u_3^i)$, pre $i=1,\dots,m$. 

Chceme odhadnúť zložky vektora $x$ tak, aby čo najvierohodnejšie predpovedal solventnosť vzhľadom na naše dáta. To vedie k optimalizačnej úlohe:

\begin{align*}
	\min~&J(x)  \\
	 &x \in \mathbb{R}^4
\end{align*}

kde

\begin{equation}
	J(x) = -\sum_{i=1}^{m}v^i \ln \left(g\left(x^Tu^i\right)\right) 
			+ (1-v^i) \ln \left(1 - g\left(x^Tu^i\right)\right) \label{eq:objf}
\end{equation}

Z predpisu funkcie si môžeme všimnúť, že suma nadobúda záporné hodnoty, čiže $J(x)$ nadobúda kladné hodnoty. Taktiež si môžeme všimnúť, že ak je klient $i$ solventný, čiže $v^i=1$ a pre nejaký vektor parametrov $x$ je hodnota $g\left(x^Tu^i\right)$ blízka nule, má to za následok \enquote{výrazné} zvyšovanie hodnoty účelovej funkcie. Podobnou logikou vidíme zvyšovanie hodnoty účelovej funkcie pre nesolventných klientov, ak pomocou vektora $x$ mu prisúdime veľkú pravdepodobnosť solventnosti hodnotou $g\left(x^Tu^i\right)$. Chceme teda nájsť taký vektor $x$, že $g\left(x^Tu^i\right)$ bude blízke 1 pre solvetného klienta a blízke 0 pre nesolventného.

\subsection{Kompaktnejší tvar účelovej funkcie}

Pre lepšiu manipuláciu a neskoršiu implementáciu si zjednodušíme tvar účelovej funkcie nasledovne:

\begin{align*}
	J(x) &= -\sum_{i=1}^{m}v^i \ln \left(g\left(x^Tu^i\right)\right) 
	+ (1-v^i) \ln \left(1 - g\left(x^Tu^i\right)\right) \\
	&= -\sum_{i=1}^{m}v^i \ln \left(\left(1 + e^{-x^Tu^i}\right)^{-1}\right)
	+ (1-v^i) \ln \left(\frac{e^{-x^Tu^i}}{1 + e^{-x^Tu^i}}\right) \\
	&= -\sum_{i=1}^{m} -v^i \ln \left(1 + e^{-x^Tu^i}\right)
	+ (1-v^i) \left(\ln \left(e^{-x^Tu^i}\right) - \ln \left(1 + e^{-x^Tu^i}\right)\right) \\
	&= -\sum_{i=1}^{m} -v^i \ln \left(1 + e^{-x^Tu^i}\right)
	- (1-v^i) x^Tu^i - (1-v^i) \ln \left(1 + e^{-x^Tu^i}\right) \\
	&= \sum_{i=1}^{m} (1-v^i) x^Tu^i + \ln \left(1 + e^{-x^Tu^i}\right)
\end{align*}

S takýmto vyjadrením funkcie $J(x)$ budeme pracovať v nasledujúcich častiach.

\subsection{Gradient účelovej funkcie}

Vyjadríme si najprv parciálnu deriváciu podľa $x_0$, potom podľa $x_{j}, j=1,2,3$, keďže tie sa správajú symetricky.

\begin{align*}
	\frac{\partial}{\partial x_0} J(x) &= \frac{\partial}{\partial x_0} \sum_{i=1}^{m} (1-v^i) x^Tu^i + \ln \left(1 + e^{-x^Tu^i}\right)\\
	&= \sum_{i=1}^{m} \frac{\partial}{\partial x_0} \left((1-v^i)(x_0 + x_1u_1^i + x_2u_2^i + x_3u_3^i) + \ln \left(1 + e^{-x^Tu^i}\right) \right)\\
	&= \sum_{i=1}^{m} (1-v^i) - \frac{e^{-x^Tu^i}}{1 + e^{-x^Tu^i}}\\
	&= \sum_{i=1}^{m} 1-v^i - \frac{1}{1 + e^{x^Tu^i}}\\
\end{align*}

\begin{align*}
	\frac{\partial}{\partial x_j} J(x) &= \frac{\partial}{\partial x_j} \sum_{i=1}^{m} (1-v^i) x^Tu^i + \ln \left(1 + e^{-x^Tu^i}\right)\\
	&= \sum_{i=1}^{m} \frac{\partial}{\partial x_j} \left((1-v^i)(x_0 + x_1u_1^i + x_2u_2^i + x_3u_3^i) + \ln \left(1 + e^{-x^Tu^i}\right) \right)\\
	&= \sum_{i=1}^{m} (1-v^i)u_j^i - u_j^i\frac{ e^{-x^Tu^i}}{1 + e^{-x^Tu^i}}\\
	&= \sum_{i=1}^{m} \left(1-v^i - \frac{1}{1 + e^{x^Tu^i}}\right)u_j^i && j=1,2,3\\
\end{align*}

Toto vieme kompaktne zapísať nasledovne:

\begin{equation*}
	\nabla J(x) = \sum_{i=1}^{m} 
	\begin{pmatrix}
		1 \\
		u_1^i \\
		u_2^i \\
		u_3^i 
	\end{pmatrix}
	\left(1-v^i - \frac{1}{1 + e^{x^Tu^i}}\right)
\end{equation*}

\end{document}